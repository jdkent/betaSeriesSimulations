% Options for packages loaded elsewhere
\PassOptionsToPackage{unicode}{hyperref}
\PassOptionsToPackage{hyphens}{url}
%
\documentclass[
]{article}
\usepackage{lmodern}
\usepackage{amssymb,amsmath}
\usepackage{ifxetex,ifluatex}
\ifnum 0\ifxetex 1\fi\ifluatex 1\fi=0 % if pdftex
  \usepackage[T1]{fontenc}
  \usepackage[utf8]{inputenc}
  \usepackage{textcomp} % provide euro and other symbols
\else % if luatex or xetex
  \usepackage{unicode-math}
  \defaultfontfeatures{Scale=MatchLowercase}
  \defaultfontfeatures[\rmfamily]{Ligatures=TeX,Scale=1}
\fi
% Use upquote if available, for straight quotes in verbatim environments
\IfFileExists{upquote.sty}{\usepackage{upquote}}{}
\IfFileExists{microtype.sty}{% use microtype if available
  \usepackage[]{microtype}
  \UseMicrotypeSet[protrusion]{basicmath} % disable protrusion for tt fonts
}{}
\makeatletter
\@ifundefined{KOMAClassName}{% if non-KOMA class
  \IfFileExists{parskip.sty}{%
    \usepackage{parskip}
  }{% else
    \setlength{\parindent}{0pt}
    \setlength{\parskip}{6pt plus 2pt minus 1pt}}
}{% if KOMA class
  \KOMAoptions{parskip=half}}
\makeatother
\usepackage{xcolor}
\IfFileExists{xurl.sty}{\usepackage{xurl}}{} % add URL line breaks if available
\IfFileExists{bookmark.sty}{\usepackage{bookmark}}{\usepackage{hyperref}}
\hypersetup{
  hidelinks,
  pdfcreator={LaTeX via pandoc}}
\urlstyle{same} % disable monospaced font for URLs
\setlength{\emergencystretch}{3em} % prevent overfull lines
\providecommand{\tightlist}{%
  \setlength{\itemsep}{0pt}\setlength{\parskip}{0pt}}
\setcounter{secnumdepth}{-\maxdimen} % remove section numbering
\usepackage[]{natbib}
\bibliographystyle{plainnat}

\date{}

\begin{document}

Results included in this manuscript come from modeling performed using
\emph{NiBetaSeries} 0.4.0 \citep{Kent2018}, which is based on
\emph{Nipype} 1.1.9 \citep{Gorgolewski2011, Gorgolewski2018}.

\hypertarget{beta-series-modeling}{%
\subsubsection{Beta Series Modeling}\label{beta-series-modeling}}

Least squares- separate (LSS) models were generated for each event in
the task following the method described in \citet{Turner2012a}, using
Nistats 0.0.1b.\\
Prior to modeling, preprocessed data were masked,and mean-scaled over
time. For each trial, preprocessed data were subjected to a general
linear model in which the trial was modeled in its own regressor, while
all other trials from that condition were modeled in a second regressor,
and other conditions were modeled in their own regressors. Each
condition regressor was convolved with a ``glover'' hemodynamic response
function for the model.\\
In addition to condition regressors, white\_matter, csf, cosine00,
cosine01, cosine02, cosine03, cosine04, cosine05, cosine06, cosine07,
cosine08, cosine09, cosine10, cosine11, cosine12 and a high-pass filter
of 0.0078125 Hz (implemented using a cosine drift model) were included
in the model. AR(1) prewhitening was applied in each model to account
for temporal autocorrelation.

After fitting each model, the parameter estimate (i.e., beta) map
associated with the target trial's regressor was retained and
concatenated into a 4D image with all other trials from the same
condition, resulting in a set of N 4D images where N refers to the
number of conditions in the task. The number of volumes in each 4D image
represents the number of trials in that condition.

\hypertarget{atlas-connectivity-analysis}{%
\subsubsection{Atlas Connectivity
Analysis}\label{atlas-connectivity-analysis}}

The beta series 4D image for each condition in the task was subjected to
an ROI-to-ROI connectivity analysis to produce a condition-specific
correlation matrix. The correlation coefficient estimator used for this
step was empirical covariance, as implemented in Nilearn 0.4.2
\citep{Abraham2014}. Correlation coefficients were converted to
normally-distributed z-values using Fisher's r-to-z conversion
\citep{Fisher1915}. Figures for the correlation matrices were generated
with Matplotlib 2.2.4 \citep{Hunter2007} and MNE-Python 0.18.2
\citep{Gramfort2013, Gramfort2014}.

\hypertarget{software-dependencies}{%
\subsubsection{Software Dependencies}\label{software-dependencies}}

Additional libraries used in the NiBetaSeries workflow include
\emph{Pybids} 0.9.4 \citep{Yarkoni2019}, \emph{Niworkflows} 0.10.4,
\emph{Nibabel} 2.3.3, \emph{Pandas} 0.24.2 \citep{McKinney2010}, and
\emph{Numpy} 1.14.5 \citep{VanDerWalt2011, Oliphant2006}.

\hypertarget{copyright-waiver}{%
\subsubsection{Copyright Waiver}\label{copyright-waiver}}

The above boilerplate text was automatically generated by NiBetaSeries
with the express intention that users should copy and paste this text
into their manuscripts \emph{unchanged}. It is released under the
\href{https://creativecommons.org/publicdomain/zero/1.0/}{CC0} license.

\hypertarget{references}{%
\subsubsection{References}\label{references}}

  \bibliography{/Users/jdkent/miniconda3/envs/betaseries\_simulation/lib/python3.6/site-packages/nibetaseries/data/references.bib}

\end{document}
